\documentclass[12pt]{article}
\usepackage[utf8]{inputenc}
\usepackage{microtype} %improve general appearance

% Document style
\usepackage[
a4paper,
width=150mm,
top=25mm,
bottom=25mm,
bindingoffset=6mm
]{geometry}

% Headers and footers:
%\usepackage{fancyhdr}
%\pagestyle{fancy}
%\fancyhead{}
%\fancyhead[RO,LE]{\slshape\nouppercase{\rightmark}}%{ML and data analysis for BehNeuro}%
%\fancyfoot{}
%\fancyfoot[LE,RO]{\thepage}
%\fancyfoot[LO,CE]{Chapter \thechapter}
%\fancyfoot[CO,RE]{Avgoustinos Vouros}
%\renewcommand{\headrulewidth}{0.4pt} %line thickness: head
%\renewcommand{\footrulewidth}{0.4pt} %line thickness: foot

% Graphics
\usepackage{graphicx,soul}
\usepackage[dvipsnames,table]{xcolor} %highlight text
\graphicspath{ {images/} } %folder under which the graphics are
\usepackage[labelfont=bf,font=footnotesize]{caption}
\usepackage{subcaption}
\usepackage[most]{tcolorbox} %coloured and framed text boxes
\addtolength{\belowcaptionskip}{-5pt}

% References
\usepackage[square,authoryear,sort]{natbib}
\bibliographystyle{agsm}    %Harvard style
\renewcommand\cite{\citep}  %\cite = \citep

% Referencing commands
\usepackage[]{hyperref}
\hypersetup{
	colorlinks=true,
	linkcolor=black,
	urlcolor=blue,
	bookmarksnumbered=true,     
	bookmarksopen=true,         
	bookmarksopenlevel=1,       
	colorlinks=true,            
	pdfstartview=Fit,           
	pdfpagemode=UseOutlines,
	pdfpagelayout=TwoPageRight
}

% Table of Contents
\setcounter{tocdepth}{4}
\setcounter{secnumdepth}{4}
\usepackage[titletoc]{appendix}
\usepackage[nottoc,notlot,notlof]{tocbibind}

% Chapter
%\usepackage{titlesec}
%\titleformat{\chapter}{\normalfont\huge}{\huge{\bf{Chapter %\thechapter:}}}{20pt}{\huge\bf}

% Dummy
\usepackage{lipsum} %text

% Empty page
%\usepackage{afterpage}
%\newcommand\blankpage{%
%	\null
%	\thispagestyle{empty}%
	%\addtocounter{page}{-1}%
%	\newpage
%}

% Tables
\usepackage{multicol}
\usepackage{multirow}
\usepackage{tabu}
\usepackage{array}
%\usepackage{booktabs}
\definecolor{newcolor}{rgb}{.8,.349,.1}

% Maths symbols and fonts
\usepackage{mathtools,amsmath,amsfonts,amssymb,amsthm,bbm,mathbbol}
% Algorithms
\usepackage[ruled,vlined]{algorithm2e}
% Caligraphic characters
\usepackage{calrsfs}
% Subindexes
\usepackage{stackengine}

% Containers and other
\usepackage{float} %containers that cannot be broken
%\usepackage{enumitem} %list environments
\usepackage[shortlabels]{enumitem}
\usepackage{pdfpages} %attach PDF documents
\usepackage{indentfirst} %indent the firsts paragraphs

% Extra configs
%...
\usepackage{setspace}

%\newenvironment{changemargins}{
%	\newgeometry{right=5mm, left=5mm}
%j}

% Extra commands
\newcommand\tab[1][1cm]{\hspace*{#1}}  %the \tab command
\usepackage{cancel} %the \cancel{math} command
\DeclarePairedDelimiter{\abs}{\lvert}{\rvert} %the \abs{math} command
%list of special caligraphic characters
\DeclareMathAlphabet{\pazocal}{OMS}{zplm}{m}{n}
\newcommand{\cA}{\mathcal{A}}
\newcommand{\CA}{\pazocal{A}}
\newcommand{\cB}{\mathcal{B}}
\newcommand{\CB}{\pazocal{B}}
\newcommand{\cC}{\mathcal{C}}
\newcommand{\CC}{\pazocal{C}}
\newcommand{\cD}{\mathcal{D}}
\newcommand{\CD}{\pazocal{D}}
\newcommand{\cE}{\mathcal{E}}
\newcommand{\CE}{\pazocal{E}}
\newcommand{\cF}{\mathcal{F}}
\newcommand{\CF}{\pazocal{F}}
\newcommand{\cG}{\mathcal{G}}
\newcommand{\CG}{\pazocal{G}}
\newcommand{\cH}{\mathcal{H}}
\newcommand{\CH}{\pazocal{H}}
\newcommand{\cI}{\mathcal{I}}
\newcommand{\CI}{\pazocal{I}}
\newcommand{\cJ}{\mathcal{J}}
\newcommand{\CJ}{\pazocal{J}}
\newcommand{\cK}{\mathcal{K}}
\newcommand{\CK}{\pazocal{K}}
\newcommand{\cL}{\mathcal{L}}
\newcommand{\CL}{\pazocal{L}}
\newcommand{\cM}{\mathcal{M}}
\newcommand{\CM}{\pazocal{M}}
\newcommand{\cN}{\mathcal{N}}
\newcommand{\CN}{\pazocal{N}}
\newcommand{\cO}{\mathcal{O}}
\newcommand{\CO}{\pazocal{O}}
\newcommand{\cP}{\mathcal{P}}
\newcommand{\CP}{\pazocal{P}}
\newcommand{\cQ}{\mathcal{Q}}
\newcommand{\CQ}{\pazocal{Q}}
\newcommand{\cR}{\mathcal{R}}
\newcommand{\CR}{\pazocal{R}}
\newcommand{\cS}{\mathcal{S}}
\newcommand{\CS}{\pazocal{S}}
\newcommand{\cT}{\mathcal{T}}
\newcommand{\CT}{\pazocal{T}}
\newcommand{\cU}{\mathcal{U}}
\newcommand{\CU}{\pazocal{U}}
\newcommand{\cV}{\mathcal{V}}
\newcommand{\CV}{\pazocal{V}}
\newcommand{\cW}{\mathcal{W}}
\newcommand{\CW}{\pazocal{W}}
\newcommand{\cX}{\mathcal{X}}
\newcommand{\CX}{\pazocal{X}}
\newcommand{\cY}{\mathcal{Y}}
\newcommand{\CY}{\pazocal{Y}}
\newcommand{\cZ}{\mathcal{Z}}
\newcommand{\CZ}{\pazocal{Z}}



\begin{document}
	
\section{Notations}

\noindent Unless specified otherwise inside the text,
\begin{itemize}
	\item The vector notation $x_{i:} = [x_{i1}, x_{i2}, \dots, x_{ip}]$ specifies the i-th element of the matrix containing the data set $\CX$ consisting of $n$ observations and $p$ dimensions (or features or data attributes). $j$ is an index on the $p$ dimensions.
	\item Given $K$ number of groups (clusters) $\CC = \{c_1, c_2, \dots, c_K \}$, with $n_1, n_2, \dots, n_K$ number of elements in each group respectively ($n$ without an index will refer to the number of elements of the whole data set), the vector notation $m_{k:} = [m_{k1}, m_{k2}, \dots, m_{kp}]$ specifies the k-th group center (centroid). This group center is the mean of the data in the group
	\item The notation $\mu_{1:}$ refers to the global center of the data set, which is unique, and $\mu_{1:} = [\mu_{11}, \mu_{12}, \dots, \mu_{1p}]$.
	\item Given $\mathsf{K}$ class labels $\ell$ 	then $n_{\ell}$
	are the number of elements belonging to each class and $n_{\ell}^{(k)}$ is the number of elements of class $\ell$ belonging to cluster $k$.
	\item The letters $w$ and $a$ are reserved to specify the weights of each dimension, i.e. $w_{1}, w_{2}, \dots, w_{p}$ and $a_{1}, a_{2}, \dots, a_{p}$.
	\item The stylized letter $\mathbb{k}$ is used to indicate the $\mathbb{k}$-fold cross validation.
	\item The notation, 
	\begin{align}\nonumber
	\sum_{\binom{i=1}{x_{i:} \in c_k}}^{n_k}x_{i:} = \sum_{\binom{i=1}{x_{i:} \in c_k}}^{n_k} \sum_{j=1}^{p} x_{ij}
	\end{align}
	specifies a summation of all the $p$-dimensional data points $x_{i:}$, $i = 1 \dots n_k$ which belong to the k-th group ($x_{i:} \in c_k$). 
	\item 
	\begin{itemize}
		\item WCSS = Within Clusters Sum of Squares.
		\item BCSS = Between Clusters Sum of Squares. 
	\end{itemize}	
\end{itemize}

\cleardoublepage

\section{K-Means}

\noindent \textbf{Objective function: Minimize WCSS \cite{jain2010data}.}
\begin{align}\label{kmeans1}
\CJ_{kmeans} &= \sum_{k=1}^{K}\sum_{\binom{i=1}{x_{i:} \in c_k}}^{n_k} \sum_{j=1}^{p} (x_{ij}-m_{kj})^2	
\end{align}	

\noindent \textbf{Objective function: Maximize BCSS \cite{witten2010framework}.}
\begin{align}\label{kmeans2}
\CJ_{BCSS} &= \sum_{i=1}^{n}\sum_{j=1}^{p}(x_{ij}-{\mu}_{1j})^2 - \sum_{k=1}^{K}\sum_{\binom{i=1}{x_{i:} \in c_k}}^{n_k}\sum_{j=1}^{p}(x_{ij}-m_{kj})^2	
\end{align}	

\noindent \textbf{Equivalent formulas \cite{witten2010framework}.}

\begin{align}\label{prrof1a}
WCSS = 
\sum_{k=1}^{K} \sum_{\binom{i=1}{x_{i:} \in c_k}}^{n_k} \sum_{j=1}^{p} (x_{ij}-m_{kj})^2	
= \sum_{k=1}^{K} \frac{1}{2n_k}\sum_{\binom{i=1}{x_{i:} \in c_k}}^{n_k} \sum_{\binom{i'=1}{x_{i':} \in c_k}}^{n_k} \sum_{j=1}^{p} (x_{ij}-x_{i'j})^2
\end{align}	

\noindent \textbf{Minimization of Eq. (\ref{kmeans1}) leads to cluster centroids.}

\begin{align}
\frac{\partial \CJ_{kmeans}}{\partial m_{k'j'}} = 0 \Rightarrow 
&\frac{\partial}{\partial m_{k'j'}} \sum_{k=1}^{K}\sum_{\binom{i=1}{x_{i:} \in c_k}}^{n_k} \sum_{j=1}^{p} (x_{ij}-m_{kj})^2 = 0 \Rightarrow 
&m_{k'j'} = \frac{1}{n_{k'}}\sum_{\binom{i=1}{x_{i:} \in c_{k'}}}^{n_{k'}} x_{ij'}
\end{align}

\noindent\textbf{K-Means algorithms.}
\begin{itemize}
	\item Lloyd's K-Means \cite{lloyd1982least, jain2010data}
	\item MacQueen K-Means \cite{macqueen1967some}
	\item Hartigan-Wong K-Means \cite{hartigan1979algorithm,slonim2013hartigan}
\end{itemize}





\clearpage


\begin{center}
	\begin{tcolorbox}[breakable,colback=white!100!white,colframe=black!100!black]
		\noindent\textbf{Lloyd's K-Means algorithm}
		\begin{enumerate}
			\item Initialise $K$ initial centroids $M = \{m_{1j}, \dots, m_{Kj}\}$ using some initialisation method. 
			
			\item Assign each data point to cluster $c_{k^*}$ so that,
			\begin{align}\nonumber
			k^* &= \underset{k}{argmin}\bigg\{ \sum_{j=1}^{p} (x_{ij}-m_{kj})^2    \bigg\}
			\end{align}	
			
			\item Recompute the cluster centroids,
			\begin{align}\nonumber
			m_{kj} = \frac{1}{n_k}\sum_{\binom{i=1}{x_{i:} \in c_k}}^{n_k} x_{ij}
			\end{align}	
			
			\item Go to step 2 until converge.						
			
		\end{enumerate}	
		The algorithm returns the final clusters (centroids and element assignments).
	\end{tcolorbox}
\end{center}

\begin{center}
	\begin{tcolorbox}[breakable,colback=white!100!white,colframe=black!100!black]
		\noindent\textbf{Lloyd's K-Means algorithm}
		\begin{enumerate}
			\item Initialise $K$ initial centroids using some initialisation method. 
			
			\item Assign each data point to its nearest centroid. 
			
			\item Recompute the cluster centroids by taking the mean of the data points belonging to each cluster.
			
			\item Go to step 2 until converge.						
			
		\end{enumerate}	
		The algorithm returns the final clusters (centroids and element assignments).
	\end{tcolorbox}
\end{center}

\cleardoublepage


\begin{center}
	\setlength\abovedisplayskip{0pt}
	\begin{tcolorbox}[breakable,colback=white!100!white,colframe=black!100!black]
		\noindent\textbf{Hartigan-Wong's K-Means algorithm}
		\begin{enumerate}
			\item Initialise $K$ initial centroids $M = \{m_{1j}, \dots, m_{Kj}\}$ using some initialisation method. 
			
			\item Assign each data point to cluster $k'$ so that,
			\begin{align}\nonumber
			k' &= \underset{k}{argmin}\bigg\{ \sum_{j=1}^{p} (x_{ij}-m_{kj})^2    \bigg\}
			\end{align}	
			
			\item Recompute the cluster centroids,
			\begin{align}\nonumber
				m_{kj} = \frac{1}{n_k}\sum_{\binom{i=1}{x_{i:} \in c_k}}^{n_k} x_{ij}
			\end{align}	
			
			\item Set an indicator $s=0$.		
			
			\item For each data point $x_{i:}$:
			
			\begin{enumerate}
				\item Remove it from its cluster $c_{k'}$ and compute $WCSS_{k'}$. Set $ WCSS_{min} = WCSS_{k'}$.
				\item For each cluster $c_{t} \neq c_{k'}$, $t = 1,\cdots, K$
				\begin{enumerate}
					\item Assign it to $c_{t}$ and compute $WCSS_{t}$
					\item If $WCSS_{t} < WCSS_{min}$, $WCSS_{min}=WCSS_{t}$ and assign $x_{i:}$ to cluster $c_{t}$.
				\end{enumerate}
				\item If $WCSS_{min} \neq WCSS_{k'}$, set $s=1$ and update $m_{k':}$ and $m_{min:}$. Else $WCSS_{min} = WCSS_{k'}$ assign $x_{i:}$ to its original cluster $c_{k'}$. 
			\end{enumerate}	
			
			\item If $s=1$, set $s=0$ and go to step 5 else terminate
			
		\end{enumerate}	
		The algorithm returns the final clusters (centroids and element assignments).
	\end{tcolorbox}
\end{center}

\begin{center}
	\begin{tcolorbox}[breakable,colback=white!100!white,colframe=black!100!black]
		\noindent\textbf{Hartigan-Wong's K-Means algorithm}
		\begin{enumerate}
			\item Initialise $K$ initial centroids using some initialisation method. 
			
			\item Assign each data point to its nearest centroid. 
			
			\item Recompute the cluster centroids by taking the mean of the data points belonging to each cluster.
			
			\item For each data point $x_{i:}$:
			
			\begin{enumerate}
				\item Remove it from its cluster and compute the WCSS of that cluster.
				\item Compute the WCSS of each other cluster if $x_{i:}$ was assigned them.
				\item Assign $x_{i:}$ to the cluster with the minimum WCSS.
				\item Update the old and the new cluster centroids of $x_{i:}$.
			\end{enumerate}
			
			\item If no data points were changed clusters terminate else go to step 4.
			
		\end{enumerate}	
		The algorithm returns the final clusters (centroids and element assignments).
	\end{tcolorbox}
\end{center}

\cleardoublepage

\begin{center}
	\begin{tcolorbox}[breakable,colback=white!100!white,colframe=black!100!black]
		\noindent\textbf{MacQueen's K-Means algorithm}
		\begin{enumerate}
			\item Initialise $K$ initial centroids $M = \{m_{1j}, \dots, m_{Kj}\}$ using some initialisation method. 
			
			\item Assign each data point to cluster $c_{k^*}$ so that,
			\begin{align}\nonumber
			k^* &= \underset{k}{argmin}\bigg\{ \sum_{j=1}^{p} (x_{ij}-m_{kj})^2    \bigg\}
			\end{align}	
			
			\item Recompute the cluster centroids,
			\begin{align}\nonumber
			m_{kj} = \frac{1}{n_k}\sum_{\binom{i=1}{x_{i:} \in c_k}}^{n_k} x_{ij}
			\end{align}	
			
			\item Set an indicator $s=0$.
			
			\item For each data point $x_{i:}$:
			
			\begin{enumerate}
				\item Assign it from cluster $c_{k'}$ to cluster $c_{k^*}$ so that,
				\begin{align}\nonumber
					k^* &= \underset{k}{argmin}\bigg\{ \sum_{j=1}^{p} (x_{ij}-m_{kj})^2    \bigg\}
				\end{align}	
				\item If $c_{k'} \neq c_{k^*}$ update $m_{k':}$ and $m_{*:}$ and set $s=1$.
			\end{enumerate}
			
			\item If $s=1$, set $s=0$ and go to step 5 else terminate					
			
		\end{enumerate}	
		The algorithm returns the final clusters (centroids and element assignments).
	\end{tcolorbox}
\end{center}

\begin{center}
	\begin{tcolorbox}[breakable,colback=white!100!white,colframe=black!100!black]
		\noindent\textbf{MacQueen's K-Means algorithm}
		\begin{enumerate}
			\item Initialise $K$ initial centroids using some initialisation method. 
			
			\item Assign each data point to its nearest centroid. 
			
			\item Recompute the cluster centroids by taking the mean of the data points belonging to each cluster.
			
			\item For each data point $x_{i:}$:
			
			\begin{enumerate}
				\item Assign it to its nearest centroid.
				\item Update the old and the new cluster centroids of $x_{i:}$.
			\end{enumerate}
			
			\item If no data points were changed clusters terminate else go to step 4.
			
		\end{enumerate}	
		The algorithm returns the final clusters (centroids and element assignments).
	\end{tcolorbox}
\end{center}

\cleardoublepage


\section{K-Medians}

\noindent \textbf{Objective function \cite{aggarwal2014data}.}

\begin{align}\label{kmed1}
\CJ_{kmedians} = \sum_{k=1}^{K}\sum_{\binom{i=1}{x_{i:} \in c_k}}^{n_k} \sum_{j=1}^{p} \abs{x_{ij}-m_{kj}}
\end{align}	

\begin{center}
	\begin{tcolorbox}[breakable,colback=white!100!white,colframe=black!100!black]
		\noindent\textbf{K-Medians algorithm}
		\begin{enumerate}
			\item Initialise $K$ initial centroids $M = \{m_{1j}, \dots, m_{Kj}\}$ using some initialisation method. 
			
			\item Assign each data point to cluster $k^*$ so that,
			\begin{align}\nonumber
			k^* &= \underset{k}{argmin}\bigg\{ \sum_{j=1}^{p} (x_{ij}-m_{kj})^2    \bigg\}
			\end{align}	
			
			\item Recompute the cluster centroids,
			\begin{align}\nonumber
				m_{kj} = 
				\left\{\begin{matrix}
				x_{ij} &, & i = (n_{k}+1)/2 &\\ 
				(x_{ij} + x_{i'j})/2 &, & i = n_{k}/2 ,& i' = (n_{k}+1)/2
				\end{matrix}\right.
			\end{align}	
			
			\item Go to step 2 until converge.		
			
		\end{enumerate}	
		The algorithm returns the final clusters (centroids and element assignments).
	\end{tcolorbox}
\end{center}

\begin{center}
	\begin{tcolorbox}[breakable,colback=white!100!white,colframe=black!100!black]
		\noindent\textbf{K-Medians algorithm}
		\begin{enumerate}
			\item Initialise $K$ initial centroids using some initialisation method. 
			
			\item Assign each data point to its nearest centroid. 
			
			\item Recompute the cluster centroids by taking the median of the data points belonging to each cluster.
			
			\item Go to step 2 until converge.						
			
		\end{enumerate}	
		The algorithm returns the final clusters (centroids and element assignments).
	\end{tcolorbox}
\end{center}
	
\cleardoublepage	
	

\section{Geometric K-Medians}

\noindent \textbf{Objective function \cite{whelan2015understanding}.}
\begin{align}\label{kmed2}
\CJ_{gkmedians} = \sum_{k=1}^{K}\sum_{\binom{i=1}{x_{i:} \in c_k}}^{n_k} \bigg{|}\sum_{j=1}^{p}(x_{ij}-m_{kj})\bigg{|}
\end{align}	

\noindent \textbf{Cluster centroids.}	
\begin{align}\label{kmed2obj}
\frac{\partial \CJ_{gkmedians}}{\partial m_{k'j'}} = \frac{\partial}{\partial m_{k'j'}} \sum_{k=1}^{K}\sum_{\binom{i=1}{x_{i:} \in c_k}}^{n_k} \bigg{|}\sum_{j=1}^{p}(x_{ij}-m_{kj})\bigg{|} = 0 
\Rightarrow
m_{k'j'} = \frac{\sum_{\binom{i=1}{x_{i:} \in c_{k'}}}^{n_{k'}} \frac{x_{ij'}}{\sqrt{(x_{ij'}-m_{k'j'})^2}}}{\sum_{\binom{i=1}{x_{i:} \in c_{k'}}}^{n_{k'}}\frac{1}{\sqrt{(x_{ij'}-m_{k'j'})^2}}}
\end{align}	
	
\begin{center}
	\begin{tcolorbox}[breakable,colback=white!100!white,colframe=black!100!black]
		\noindent\textbf{Weiszfeld's algorithm}\label{algoweis}
		\begin{enumerate}
			\item Initialise $K$ initial centroids $M = \{m_{1j}, \dots, m_{Kj}\}$ using some initialisation method. 
			
			\item Assign each data point to cluster $k^*$ so that,
			\begin{align}\nonumber
			k^* &= \underset{k}{argmin}\bigg\{ \sum_{j=1}^{p} (x_{ij}-m_{kj})^2    \bigg\}
			\end{align}		
			
			\item Recompute the cluster centroids using the Weiszfeld's formula,
			
			\begin{enumerate}
				\item For each cluster $k$ and dimension $j$:
				\item Initialise the $k$-th centroid,
				\begin{align}\nonumber
				m_{kj} = \frac{1}{n_k}\sum_{\binom{i=1}{x_{i:} \in c_k}}^{n_k} x_{ij}
				\end{align}	
				\item Update the centroid estimation, $m_{kj}^{(l)} = \frac{\sum_{\binom{i=1}{x_{i:} \in c_k}}^{n_k} \frac{x_{ij}}{
						\sqrt{		\sum_{j=1}^{p}
							(x_{ij}-m_{kj}^{(l)})^2}}}{\sum_{\binom{i=1}{x_{i:} \in c_k}}^{n_k}\frac{1}{
						\sqrt{		\sum_{j=1}^{p}
							(x_{ij}-m_{kj}^{(l)})^2}}}$, \\$l = 1, \dots, n_k$
			\end{enumerate}			
			
			\item Go to step 2 until converge.	
			
		\end{enumerate}	
		The algorithm returns the final clusters (centroids and element assignments).
	\end{tcolorbox}
\end{center}	


\cleardoublepage


\section{Sparse K-Means}

\noindent \textbf{Objective function \cite{witten2010framework}.} 
	
\begin{align}\nonumber
\CJ_{skmeans} =& \sum_{i=1}^{n}\sum_{j=1}^{p} w_{j}(x_{ij}-{\mu}_{1j})^2 - \sum_{k=1}^{K}\sum_{\binom{i=1}{x_{i:} \in c_k}}^{n_k}\sum_{j=1}^{p} w_{j}(x_{ij}-m_{kj})^2  \Rightarrow\\ \label{skmeansopt2}
\CJ_{skmeans} =&  \sum_{j=1}^{p} w_{j}\gamma_j \hphantom{x}\text{, with}\hphantom{x} 	\gamma_j = \sum_{i=1}^{n} (x_{ij}-{\mu}_{1j})^2 - \sum_{k=1}^{K}\sum_{\binom{i=1}{x_{i:} \in c_k}}^{n_k}(x_{ij}-m_{kj})^2\\
\nonumber
\text{subject to}& \hphantom{xxxxx} \sum_{j=1}^{p} w_{j}^{2} \leq 1 \text{,}\hphantom{x} \sum_{j=1}^{p} \abs{w_{j}} \leq s \text{,}\hphantom{x} w_{j} \geq 0\hphantom{x} \forall j 
\end{align}	

\noindent \textbf{Weights optimization.} 

\begin{align}\nonumber
\underset{w_j}{maximize}& \bigg\{ \sum_{j=1}^{p} w_{j}\gamma_j \bigg\} \hphantom{x,}
\text{subject to} \hphantom{xxx} \sum_{j=1}^{p} w_{j}^{2} \leq 1 \text{,}\hphantom{x} \sum_{j=1}^{p} \abs{w_{j}} \leq s \text{,}\hphantom{x} w_{j} \geq 0\hphantom{x} \forall j \Rightarrow
\end{align}	

\begin{align}\label{skmeansopt3}
w_{j} = \frac{sign(\gamma_{j})(\vphantom{x} \abs{\gamma_{j}}-\Delta)_{+}}{
	\sqrt{\sum_{j'=1}^{p}\big{(}sign(\gamma_{j'})(\vphantom{x} \abs{\gamma_{j'}}-\Delta)\big{)}^2}
}
\end{align}	

\noindent where $x_+ = \mathcal{H}(x)\cdot x$, $\mathcal{H}$ is the Heaviside function and $x \in \mathbb{R}$. We assume that $\gamma_j$ has a unique maximum and that $1 \leq s \leq \sqrt{p}$.

\begin{center}
	\begin{tcolorbox}[breakable,colback=white!100!white,colframe=black!100!black]
		\noindent\textbf{Sparse K-Means algorithm \cite{witten2010framework}}
		\begin{enumerate}
			\item Initialise $K$ initial centroids $M = \{m_{1j}, \dots, m_{Kj}\}$ using some initialisation method and the feature weights as $w_{1} = \dots = w_{p} = \frac{1}{\sqrt{p}}$.
			\item Holding the weights fixed, maximize \ref{skmeansopt2} with respect to $M$. This can be achieved by performing K-Means on the scaled data, i.e. multiply each feature $j$ with $\sqrt{w_j}$.	
			\item Holding $M$ fixed optimize equation \ref{skmeansopt2} with respect to the weights applying the proposition given in equation \ref{skmeansopt3}. Choose $\Delta = 0$ if that leads to $\sum_{j=1}^{p} \abs{w_{j}} \leq s$, otherwise find $\Delta > 0$ that results in $\sum_{j=1}^{p} \abs{w_{j}} = s$. To find $\Delta$ the Bisection algorithm can be used.
			\item Go to step 2 until the convergence criterion in equation \ref{skmeanconv}
			\begin{align}\label{skmeanconv}
			\frac{ \sum_{j=1}^{p}\abs{w_{j}^{r}-w_{j}^{r-1}} }{ w_{j}^{r-1} } < 10^{-4} \hphantom{x}\text{, if}\hphantom{x}r > 1
			\end{align}	
			where $r$ refers to the current iteration, and $w_{j}^{r-1}$ to the weights of the previous iteration.
		\end{enumerate}	
		The algorithm returns the final clusters (centroids and elements) and the weight of each feature.		
		
		\vspace{1.0cm}
		
		\noindent\textbf{Bisection algorithm}
		\begin{enumerate}
			\item Assume $lim_1 < \Delta < lim_2$, $lim_1 = 0$ and $lim_2 = max(\gamma_1,...,\gamma_p)$
			
			\item Compute $\Delta = \frac{lim_1+lim_2}{2}$ and set
			\begin{align}\nonumber
			\begin{Bmatrix}
			lim_2 = \Delta
			&,\hphantom{x  } \text{if} \hphantom{x } \sum_{j=1}^{p} \abs{w_{j}} < s \\
			lim_1 = \Delta
			&,\hphantom{x  } \text{if} \hphantom{x } \sum_{j=1}^{p} \abs{w_{j}} \geq s
			\end{Bmatrix}
			\end{align}	
			
			\item If $lim_2-lim_1 \geq 10^{-4}$ go to step 2.
		\end{enumerate}
	\end{tcolorbox}
\end{center}

%\begin{align}\label{pckmeans}
%\CJ_{pckm} &= \sum_{k=1}^{K}  \sum_{\binom{i=1}{x_{i:} \in c_k}}^{n_k} \bigg( \sum_{j=1}^{p} (x_{ij}-m_{kj})^2 + 
%\nonumber\\&
%\sum_{(x_{i:})ML(x_{i':})} \sum_{j=1}^{p} %b_{x_i,x_{i'}} (x_{ij}-x_{i'j})^2 \hphantom{x}\mathbbm{1} \big[(x_{i:})\cancel{ML}(x_{i':})\big] + 
%\nonumber\\&
%\sum_{(x_{i:})CL(x_{i':})} \sum_{j=1}^{p} \bar{b}_{x_i,x_{i'}} \big((x_{Ij}-x_{I'j})^2 - (x_{ij}-x_{i'j})^2\big) \hphantom{x}\mathbbm{1} \big[(x_{i:})\cancel{CL}(x_{i':})\big] \bigg)
%\end{align}	
	

\cleardoublepage
\section*{Publications \& Online Material}
\begin{itemize}
	\item \cite{vouros2019empirical}
	\item \cite{vouros2020semi}
\end{itemize}	
\bibliography{references}	
\end{document}

